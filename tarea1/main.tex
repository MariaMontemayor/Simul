\documentclass[12pt]{amsart}

\addtolength{\hoffset}{-2.25cm}
\addtolength{\textwidth}{4.5cm}
\addtolength{\voffset}{-2.5cm}
\addtolength{\textheight}{5cm}
\setlength{\parskip}{0pt}
\setlength{\parindent}{15pt}
\usepackage{listings}
\usepackage{amsthm}
\usepackage{amsmath}
\usepackage[sort&compress, numbers]{natbib}
\usepackage{amssymb}
\usepackage[colorlinks = true, linkcolor = black, citecolor = black, final]{hyperref}


\usepackage{graphicx}
\usepackage{multicol}
\usepackage{ marvosym }
\newcommand{\ds}{\displaystyle}


\pagestyle{myheadings}

\setlength{\parindent}{0in}

\pagestyle{empty}

\begin{document}

\thispagestyle{empty}

{\scshape Simulación} \hfill {\scshape \Large Tarea 1: Movimiento Browniano} \hfill {\scshape 24/Feb/2021}
\author{C. María Montemayor Palos}
\maketitle

\hrule
\hrule
\bigskip

\bigskip

\section{Objetivo}
El objetivo de esta práctica es examinar sistemáticamente los efectos de la dimensión en el tiempo de regreso al origen del movimiento Browniano para dimensiones 1 a 5 en incrementos lineales de uno, variando el número de pasos de la caminata como potencias de dos con exponente de 4 a 9 en incrementos lineales de uno, realizando 30 repeticiones del experimento para cada combinación.

\section{Metodología}
Se utilizó el programa R versión 4.0.4 \cite{R} para Windows para llevar a cabo el movimiento Browniano. Se generó un código para evaluar el tiempo del regreso al origen de una partícula que se mueve al azar, tomando como código base la rutina de caminata \cite{Dra.Elisa} de la partícula.

Se variaron los pasos de la caminata con potencias de 2 con exponentes de 4 a 9 programando 30 repeticiones para cada experimento en dimensiones de 1 a 5. Se tomó como apoyo el repositorio de CrisAE. \cite{CrisAE}

\section{Código}
\begin{lstlisting}
#tarea1.R

library (parallel)
origenes=data.frame()
reg=data.frame()

for (expo in 4:9){
  caminata <- 2**expo #en pasos de 2 con exponenciales de 4 a 9
  for(dim in 1:5){ #dimensiones de 1 a 5
    for(rep in 1:30){ #repeticiones de 1 a 30
      par <- rep(0,dim)
      regreso <- FALSE
      for(t in 1:caminata){
        cambiar <- sample(1:dim,1)
        if(runif(1)<0.5){
          par[cambiar] <- par[cambiar] + 1
        }else{
          par[cambiar] <- par[cambiar]-1
        }
        if(all(par==0)){
          origenes <- rbind(origenes,c(caminata,dim,t))
          regreso=TRUE
          break
        }
      }
      if(!regreso){nreg=rbind(reg,c(caminata,dim))}
    }
  }
}

names(origenes)=c('caminata','dimension','tiempo')
names(reg)=c('caminata','dimension')

print(origenes)

\end{lstlisting}

\section{Resultados y discusión}
Para observar su efecto en el tiempo de regreso al punto de origen de la partícula, se aumentaron de manera lineal los pasos de 2 con exponentes de 4 a 9, siendo el máximo de regreso 160, como se observa en la Tabla 1.

\bigskip
\begin{center}
 \begin{tabular}{||c c c c||} 
 \hline
     & caminata & dimension & tiempo \\ [0.5ex] 
 \hline\hline
 1 & 16 & 1 & 6 \\ 
 \hline
 57 & 32 & 1 & 20 \\
 \hline
 113 & 64 & 1 & 10 \\
 \hline
 169 & 128 & 1 & 48 \\
 \hline
 254 & 256 & 1 & 160 \\ [1ex] 
 \hline
\end{tabular}
\end{center}
\bigskip
\begin{center} \caption{Tabla 1: caminata en la dimension 1 y su respectiva duración.\label{tabla:1}}
\end{center}


\bigskip
Se logra apreciar que en la tabla 2, el tiempo máximo de regreso es de 428 para la dimensión 2.

\bigskip
\begin{center}
 \begin{tabular}{||c c c c||} 
 \hline
     & caminata & dimension & tiempo \\ [0.5ex] 
 \hline\hline
 21 & 16 & 2 & 4 \\ 
 \hline
 22 & 16 & 2 & 2 \\
 \hline
 262 & 256 & 2 & 166 \\
 \hline
 266 & 256 & 2 & 144 \\
 \hline
 322 & 512 & 2 & 428 \\ [1ex] 
 \hline
\end{tabular}
\end{center}
\bigskip
\begin{center} \caption{Tabla 2: caminata en la dimension 2 y su respectiva duración.\label{tabla:2}}
\end{center}

\bigskip
En la tabla 3 el tiempo máximo de regreso es de 132 para la dimensión 3.

\bigskip
\begin{center}
 \begin{tabular}{||c c c c||} 
 \hline
     & caminata & dimension & tiempo \\ [0.5ex] 
 \hline\hline
 37 & 16 & 3 & 10 \\ 
 \hline
 82 & 32 & 3 & 10 \\
 \hline
 148 & 64 & 3 & 46 \\
 \hline
 218 & 128 & 3 & 60 \\
 \hline
 275 & 256 & 3 & 132 \\ [1ex] 
 \hline
\end{tabular}
\end{center}
\bigskip
\begin{center} \caption{Tabla 3: caminata en la dimension 3 y su respectiva duración.\label{tabla:3}}
\end{center}

\bigskip
\bigskip

El tiempo máximo de regreso es de 52 para la dimensión 4 como se aprecia en la tabla 4.

\bigskip
\begin{center}
 \begin{tabular}{||c c c c||} 
 \hline
     & caminata & dimension & tiempo \\ [0.5ex] 
 \hline\hline
 38 & 16 & 4 & 2 \\ 
 \hline
 88 & 32 & 4 & 6 \\
 \hline
 159 & 64 & 4 & 52 \\
 \hline
 224 & 128 & 4 & 6 \\
 \hline
 285 & 256 & 4 & 4 \\ [1ex] 
 \hline
\end{tabular}
\end{center}
\bigskip
\begin{center} \caption{Tabla 4: caminata en la dimension 4 y su respectiva duración.\label{tabla:4}}
\end{center}

\bigskip
Se logra apreciar que en la tabla 5, el tiempo máximo de regreso es de 6 para la dimensión 5 en la caminata 16 y 32.

\bigskip
\begin{center}
 \begin{tabular}{||c c c c||} 
 \hline
     & caminata & dimension & tiempo \\ [0.5ex] 
 \hline\hline
 43 & 16 & 5 & 6 \\ 
 \hline
 44 & 16 & 5 & 2 \\
 \hline
 45 & 16 & 5 & 2 \\
 \hline
 93 & 32 & 5 & 4 \\
 \hline
 94 & 32 & 5 & 6 \\ [1ex] 
 \hline
\end{tabular}
\end{center}
\bigskip
\begin{center} \caption{Tabla 5: caminata en la dimension 2 y su respectiva duración.\label{tabla:5}}
\end{center}

\bigskip
Se puede concluir que la partícula tarda más en regresar al punto de origen o regresa un menor número de veces cuando se encuentra en la dimensión 5, por lo tanto entre menor sea la cantidad de dimensiones, será mayor el tiempo de regreso de la partícula al punto de origen.

\bibliography{Dra.Elisa}
\bibliographystyle{plainnat}


\bigskip




\end{document}